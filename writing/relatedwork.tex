Some of the earliest work of sentiment analysis of online data was done by Turney \cite{turney2002}, Pang, Lee, and Vaithyanathan \cite{pang2002}, among others.  Pang, Lee, and Vaithyanathan highlight earlier ``knowledge-based'' sentiment classifiers that incorporated pre-selected sets of words or linguistics-based models rather than, in their words, the ``completely prior-knowledge-free supervised machine learning methods'' they employ.  They used \Na\ Bayes (NB), Maximum Entropy, and Support Vector Machines (SVMs) classifiers to great effect, getting accuracies between 77\%--83\%, depending on how they defined their feature set.  Turney, although strongly machine learning-oriented, did not have a ``completely prior-knowledge-free'' method, but built a classifier based on the mutual information between phrases and the words ``excellent'' and ``poor''.  He achieved an average accuracy of 74\%.

Sentiment analysis has exploded in the decade and a half since these initial works.  In fact, sentiment analysis is itself a business.  An informal listing of such businesses can be found on, e.g., the Quora post ``Which companies are doing real time sentiment analysis on social media?'' \cite{quora}.  A Google search for ``why does sentiment analysis matter for my business'' finds many articles from many business-related sites about sentiment analysis and why it is useful for a business.  Much progress has been made in the field, and the top-performing fully automated sentiment analyses of 2017 are far more sophisticated than those of 2002.  As just one (outdated) example, a 2008 paper from several Google associates \cite{blair2008} describes pulling aspects of a particular location or service from a comment and then rating individual aspects in an automated way.  As both sentiment analyses get more sophisticated and a greater volume and variety of reviews and information get poured into the Internet, sentiment analysis is sure to become increasingly used in the future.